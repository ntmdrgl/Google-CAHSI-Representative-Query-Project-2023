\documentclass{article}
\usepackage{algorithm}
\usepackage[noend]{algpseudocode}

\title{Pseudo Code for ``Uniform Random Sampling using 1-D Range Tree''}
\author{Nathaniel Madrigal, Alexander Madrigal}
\date{8 October, 2023}

\begin{document}

\maketitle

\begin{algorithm}
\caption{$Build1DRangeTree(P)$}
\begin{algorithmic}[1]
    \Statex $Input.$ A set of P points on a line
    \Statex $Output.$ The root of a 1-dimensional range tree
    \If{$P$ contains only one point} 
        \State Create a leaf node $v$ storing this point and store 1 as the weight of v
    \Else 
        \State Split $P$ into two subsets; one subset $P_{left}$ contains $x$-coordinate less \Statex \hspace{11.5 pt} than or equal to $x_{mid}$, the median $x$-coordinate, and the other subset \Statex \hspace{11.5 pt} $P_{right}$ contains points with $x$-coordinate larger than $x_{mid}$
        \State $v_{left} \gets Build1DRangeTree(P_{left})$
        \State $v_{right} \gets Build1DRangeTree(P_{right})$
        \State Create a node $v$ storing $x_{mid}$, make $v_{left}$ the left child of $v$, make $v_{right}$ \Statex \hspace{11.5 pt} the right child of $v$, and make the sum of $weight(v_{left})$ \Statex \hspace{11.5 pt} and $weight(v_{right})$ the weight of v
    \EndIf
    \State \textbf{return} $v$    
\end{algorithmic}
\end{algorithm}

\begin{algorithm}
\caption{$FindSplitNode(T, x, x^\prime)$}
\begin{algorithmic}[1]
    \Statex $Input.$ A 1-dimensional range tree $T$, $x$, and $x^\prime$ where $x \le x^\prime$
    \Statex $Output.$ The node $v$ where the paths to $x$ and $x^\prime$ split, or the leaf where both paths end
    \State $v \gets root(T)$
    \While{$v$ is not a leaf \textbf{and} $(x^\prime \le x_{v}$ or $x > x_{v})$}
        \If{$x^\prime \le x_{v}$} 
        \State $v \gets lc(v)$
        \Else 
        \State $v \gets rc(v)$
        \EndIf
    \EndWhile
    \State \textbf{return} $v$
\end{algorithmic}
\end{algorithm}

\begin{algorithm}
\caption{$FindCanonicalSet(T,[x:x^\prime]$}
\begin{algorithmic}[1]
    \Statex $Input.$ A 1-dimensional range tree $T$ and a range $[x:x^\prime]$
    \Statex $Output.$ A set of all canonical nodes in $T$ that lie in the range
    \State Create an empty set $C$
    \State $v_{split} \gets FindSplitNode(T, x, x^\prime)$
    \If{$v_{split}$ is a leaf}
        \State Add point $v_{split}$ to $C$ if point is in range 
    \Else
        \State (* Follow the path to $x$ and add points to the right of the path to $C$ *)
        \State $v \gets lc(v_{split})$
        \While{$v$ is not a leaf}
            \If{$x \le x_v$}
                \State Add $rc(v)$ to C
                \State $v \gets lc(v)$
            \Else 
                \State $v \gets rc(v)$
            \EndIf
        \EndWhile
        \State Add the point stored at leaf $v$ to $C$ if point is in range
        \State Similarly, follow the path to $x^\prime$, add points to the left of path to $C$, and \Statex \hspace{11.5 pt} check if point stored a the leaf where the path ends is in range and must \Statex \hspace{11.5 pt} be added to $C$
    \EndIf
    \State \textbf{return} $C$
\end{algorithmic}
\end{algorithm}

\begin{algorithm}
\caption{$UniformRandomNode(C)$}
\begin{algorithmic}[1]
    \Statex $Input.$ A set of canonical nodes $C$ with size $n$
    \Statex $Output.$ A uniform random node within the subleafs of $C$  
    \State (* Select a weighted random canonical node*)
    \For{each canonical node $c_i \in C$}
        \State Calculate a key $k_i = u_i^{1/w_i}$, where $u_i = random(0,1)$ and $w_i$ is the \Statex \hspace{11.5 pt} weight of node $c_i$
    \EndFor
    \State Store the canonical node with the greatest key in $c_{max}$
    \State (* Traverse down $c_{max}$ following the path with greatest keys *)
    \State $v \gets c_{max}$
    \While{$v$ is not a leaf}
        \State Calculate the key of left child $k_{lc} = u^{1/w_{lc}}$, where $u = random(0,1)$ and \Statex \hspace{11.5 pt} $w_{lc}$ is the weight of $lc(v)$
        \State Calculate the key of right child $k_{rc} = u^{1/w_{rc}}$, where $u = random(0,1)$ \Statex \hspace{11.5 pt} and $w_{rc}$ is the weight of $rc(v)$
        \If{$k_{rc} \le k_{lc}$}
            \State $v \gets lc(v)$
        \Else
            \State $v \gets rc(v)$
        \EndIf
    \EndWhile
    \State \textbf{return} v 
\end{algorithmic}
\end{algorithm}

\end{document}
